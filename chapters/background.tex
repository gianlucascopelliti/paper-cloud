\section{Background}

\subsection{VM-based TEEs}
%
Confidential VMs are enabled by recent technologies such as AMD
\sevsnp~\cite{AmdSevSnpWhitepaper}, Intel \ac{TDX}~\cite{intelTdxWhitepaper},
and ARM \ac{CCA}~\cite{armCCA}. While the inner workings of these mechanisms may
differ, they all allow to essentially wrap an entire VM, including its firmware,
OS, and applications, into a \ac{TEE} and also attest its authenticity to a
third party via a cryptographic protocol. This contrasts greatly with
enclave-based \acp{TEE}, such as Intel SGX, which provide an isolated execution
environment for a trusted partition of a larger untrusted user space
application. 

%As such, enclave-based \acp{TEE} are suitable for providing
%confidential computing to smaller code bases, such as cryptographic libraries,
%that can be tailored to communicate with the rest of the application and the
%host OS via secure interfaces without prohibitive development overhead.

On the other hand, VM-based \acp{TEE} are mostly transparent to the VM running
inside it except for an adaptation layer, either in the guest system software or
a paravisor~\cite{coconutSVSM}. Hence, they can provide \ac{CC} to larger
applications deployed as VMs without changing application code. Moreover,
unmodified \ac{CC} can be provided to containers via micro-VM container runtimes
such as Kata~\cite{kata}. Thus, \acp{CVM} are considered to be more suitable for
cloud-based deployments.
%
%Finally, software is easier to port between different
%\ac{CVM} back-end technologies, since adapting a VM build process to is
%conceivably more manageable than re-designing an application for different
%enclave interfaces. 
%
Nevertheless, this ease-of-use comes with an increased \ac{TCB} in the \ac{TEE}.
This not only increases the attack surface and risk for vulnerabilities, but
also complicates attestation. 

For instance, a recent study on AMD SEV found that measuring and encrypting even
moderately sized portions of the initial CVM memory leads to prohibitively large
boot times~\cite{severifast}. Also, build reproducibility becomes more complex
the more different software components are combined into the initial VM image.
Hence, measuring the complete VM along with the launch of the \ac{TEE} is
considered impractical and current \ac{CVM} architectures instead follow a
staged attestation approach that first creates a root of trust within the VM
firmware. Subsequent attestation steps are then integrated into the VM boot
process, leveraging measured/secure boot technologies to incrementally build a
chain of trust for the loaded software modules.

%However, for integration into a specific cloud platform, further configuration
%is usually required, e.g., for storage, network connectivity, or authentication
%and access control. The standard way to achieve this is to use the
%\texttt{cloud-init} tool~\cite{cloudInit} which integrates into the OS boot
%process and receives a \emph{cloud-config} file from the VM host. According to
%this configuration file, the tool configures network interfaces, create user
%accounts, set up secure communication, install applications, and run
%user-defined scripts to ready the VM for service.

\subsection{Measured Boot and Attestation}

Even without consideration for \acp{TEE}, booting a VM is a complex process that
is composed of several stages. The first components to run in such a Linux VM
are its firmware, e.g., \ac{OVMF} and bootloader, e.g., GRUB. In some cases,
such as with Direct Linux Boot~\cite{directLinuxBoot}, it is possible to skip
the bootloader altogether and boot a kernel directly from the firmware. Once
booted, the kernel extracts and mounts the initial RAM disk, e.g.,
\emph{initramfs}, and executes the contained \emph{init} script. Its sole
purpose is to mount the actual root filesystem for the VM containing the OS and
user space applications which is provided as a block device mapped into VM
memory. After mounting this filesystem, the script changes the root directory to
it and launches the OS, which in turn may start user applications.

The integrity and authenticity of the boot process is paramount to ensure the
trustworthiness of any system. Measured boot~\cite{measuredBoot} is a common
technique to produce evidence that a computer system has booted securely. While
its implementation may differ between hardware architectures, measuring the boot
process usually relies on a fixed and trusted initial firmware stage and a
tamperproof \ac{RoTM}, such as a \ac{TPM}. The initial firmware stage may
cryptographically measure itself and load subsequent boot code which, in turn,
may continue the measurement process and extend the produced measurement
results. The measurement process and storage of results, typically hashes, is
handled by the \ac{RoTM}, and the authenticity of results can be verified at a
later stage by a third party. For \acp{TEE}, the \ac{RoTM} is integrated with
the \ac{TEE} implementation, e.g., in microcode for Intel SGX, or in a secure
co-processor for AMD \sevsnp, called AMD \ac{SP}. When setting up a \ac{TEE}
instance, the untrusted system interact with the \ac{RoTM} through secure
interfaces to create trustworthy measurements of the instance. For \acp{CVM},
measured boot is established by measuring the guest firmware along with the
\ac{TEE} itself during setup. For subsequent measurements, a local \ac{RoTM}
needs to be included to the firmware, e.g., via a \ac{vTPM}. Measurement results
can then be certified by the \ac{RoTM} and sent to a remote verifier for
verification.

%Then the process for verifying the measurements of the guest boot process is the
%same as for bare-metal machines.

%Attestation is then handled via a
%challenge-response protocol over a secure channel to the \ac{TEE} hardware
%interfaces. 

% The verifier sends a challenge to the \ac{RoTM},
%which responds with the measurements and a freshness proof, e.g., by signing
%measurement results along with the nonce. Given that the \ac{RoTM} is
%tamper-proof and authenticated, the verifier is now satisfied that the boot
%process was performed correctly.


