\section{CVM Attestation in Public Clouds}
\label{sec:scenario}

Above, we have discussed the different attestation levels a verifier can achieve
when evaluating the authenticity and integrity of a deployed \ac{CVM}.  However,
who exactly plays the role of verifier depends greatly on the application
scenario and the trust relations between the involved parties.  In this work, we
focus on a scenario where a tenant wants to deploy a \ac{CVM} as guest owner on
a cloud platform, acting as the verifier in the attestation process. Afterwards,
attestation results may also be exposed to end users as
well~\cite{galanou2023revelio}. While different tenants may have different trust
relations with a given \ac{CSP} and lower attestation levels may suffice for
them in practice, we assume here that the tenant wants to reduce the required
trust in the infrastructure provider as much as possible and achieve a high
level of trustworthiness for the \ac{CVM} solution via attestation. 

Here, the measurement reports used during attestation should ideally be
verifiable independently of the \ac{CSP}. In particular, attestation reports
produced by the \ac{TEE} hardware need to be available to the verifier. In
addition, the verifier needs to be able to identify trusted software deployed in
the \ac{TEE}, i.e., they need to know the corresponding reference measurements
for the firmware, kernel, user applications, etc. A \ac{CSP} may provide these
reference values so that the authenticity of the deployed software can be
established. However, this results in a notion of verifiability that still
relies on trust in the \ac{CSP} and in the security of the deployed software. In
principle, a trusted third party such as Intel Trust
Authority~\cite{intelTrustAuthority} could certify the security of that software
and publish the corresponding reference values, but that just shifts the
required trust to a different entity.

A stronger \emph{verifiability without trust} entails that the verifier can
inspect the source code of software deployed in the \ac{TEE} and obtain the
reference measurements via a reproducible build process. Having access to the
source code allows the verifier to conduct their own security analysis of the
code and judge its trustworthiness. Of course, if the code is publicly
available, this review can also be performed by the open source community, but
this would again introduce required trust into the picture.

Overall, for our cloud \ac{CVM} scenario, the tenant wants to achieve
verifiability without trust for as high an attestation level as possible. Even
if a high \emph{nominal} AL can be achieved with the help of the \ac{CSP}, still
being
forced to trust the \ac{CSP} blindly for the measurement values and deployed
software reduces the trustworthiness. We introduce the term \emph{trustworthy
AL} to denote the maximum, effective attestation level the tenant can achieve as
a verifier without trust into the public cloud provider.