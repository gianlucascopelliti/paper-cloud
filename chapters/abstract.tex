A major drawback of cloud computing used to be the lack of confidentiality and
verifiability of computations, making it impossible to use public commercial
clouds to work with sensitive code or data. With the availability of Trusted
Execution Environments (TEEs) came the promise of enabling confidential
computations in the cloud. A number of big Cloud Service Providers (CSP) now
supports the deployment of Confidential Virtual Machines (CVMs) that can be
attested remotely, supposedly guaranteeing verifiable isolation and integrity,
and removing potentially compromised or malicious infrastructure from the
system's Trusted Computing Base (TCB). In this paper, we investigate this claim
and examine the CVM infrastructure provided by commercial CSPs regarding the
attestability of the TEE hardware and the entire CVM software stack, and
transparency regarding software provisioned by the CSP. We develop a hierarchy
of attestation levels to explain our findings and trust limitations. For the
services analysed, we observe that many attestation steps can only partially be
verified by the CVM owner. Thus, running CVMs on these CSPs' infrastructures
does not allow full TCB reduction through independently verifiable attestation
but requires trust in the CSP to deploy secure software and to truthfully report
attestation data. Complete protection from infrastructural threats is thus not
provided.
