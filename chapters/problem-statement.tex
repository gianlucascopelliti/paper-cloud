\section{Problem Statement}

This work examines trust relations involved in the \ac{CVM} deployment process.
To avoid potential confusion, we first delineate what we mean by ``trust'' and
``trustworthiness'' in this context.

Cloud service customers expect \acp{CSP} to run their cloud platform securely,
e.g., that best practices are followed to prevent compromise by outsiders,
customer data is sufficiently secured from leakage to other tenants, and
policies are in place to curtail insider threats. Depending on their specific
requirements, a tenant needs to perform a risk analysis before deploying their
software at a given cloud platform given the information they have about
it. This information is often incomplete as most \acp{CSP} run proprietary
software to operate their platforms and customers must inherently rely on the
promises of a \ac{CSP}'s marketing department for operational aspects that are
opaque to them, while also soft metrics such as reputation, track record, or
popularity may factor into their decision.

Accepting any residual risk, the tenant can then be said to \emph{trust} the
cloud platform. However, after making the decision, this trust becomes mandatory
as the customer simply has to rely on the \ac{CSP} to hold up their end of the
bargain. This trust is also perpetuated \emph{blindly} until objective evidence
surfaces that it is no longer justified.

We posit that the dual of this blind trust is \emph{trustworthiness}, which
allows for continuous risk assessment by a cloud tenant through evidence of
platform security obtained at run time. Such evidence may come in the form of
attestation reports for software deployed in \acp{TEE} together with the
possibility to independently verify the security of the TCB. The more such
run-time evidence is provided, the greater the trustworthiness of an execution
platform becomes, and the less blind trust is required.

For the \ac{CVM} case this means that the \ac{CSP} not only has to provide the
necessary infrastructure to attest the authenticity of all boot stages, but also
provide transparency for the software components provisioned by the \ac{CSP} to
allow security analysis of the remaining \ac{TCB}.

This work aims to examine this issue more closely and proposes a hierarchy of
attestation levels for the \ac{CVM} that allow to obtain increasing amounts of
runtime evidence for the integrity and authenticity of a \ac{CVM}, thus allowing
to increase trustworthiness. We then demonstrate the utility of this metric in
an analysis of popular public cloud platforms offering AMD \sevsnp{}
functionality judging the maximum level of trustworthiness achievable by them.
