\section{Introduction}
%
Cloud computing is the backbone of digitalized societies, supporting today's
cloud-native web service development paradigm. Increasingly, also industries
that traditionally operated their own IT environments, e.g., telecommunication,
manufacturing, and healthcare, are now moving to the cloud for better
scalability, manageability, and cost reduction. However, this trend is
accompanied by questions about the security and trustworthiness of
\emph{\acp{CSP}} along with regulatory concerns about privacy, data protection,
and data sovereignty: ``moving to the cloud'' involves moving large amounts of
sensitive data, along with some of the responsibility to protect it, to a third
party.

Different forms of \emph{\ac{CC}} have surfaced to address such concerns,
with \emph{\acp{TEE}}~\cite{schneider2022hardware} such as Intel SGX and
TDX, AMD SEV-SNP, or ARM CCA promising to effectively take the cloud
provider out of the \emph{\ac{TCB}}. These \acp{TEE} provide secure
compartments on cloud servers that can run applications while protecting
data in use from infrastructural threats and enabling the ``removal of even
the cloud provider from the Trusted Computing
Base''~\cite{confcon2022marketing}. At the same time they provide
\emph{attestation} mechanisms to cryptographically verify that the
unmodified application is indeed running in an authentic \ac{TEE}.  On
paper, such a design allows for a drastic reduction of customers'
dependency on cloud providers to adequately secure the cloud platform
infrastructure -- firmware, OS, and virtualization layers -- and leaving
only the hardware vendor that implements a \ac{TEE} as a root of trust.

While early \acp{TEE} for server systems embedded \emph{enclaves} into user
processes, the current market trend are \emph{\acp{CVM}} that place an entire
tenant VM into a \ac{TEE}.  While this approach increases the \ac{TCB} within
the \ac{TEE}, it simplifies the deployment of software in \acp{TEE} and reduces
system-call performance overheads~\cite{akram2020performance} relative to
enclaves, thus lowering the \ac{CC} adoption hurdle for customers. Consequently,
cloud providers such as \ac{AWS}, Microsoft Azure, or \ac{GCP} have recently
started marketing \ac{CVM} solutions as an easy fix for the security, privacy,
and trustworthiness challenges outlined above.

However, launching and attesting a \ac{CVM} is a different beast than the
enclave-based \ac{CC} approach. A VM consists of layers of system software,
including firmware, bootloader, kernel, and guest OS, as well as user space
applications, with many components usually provided by the \ac{CSP}.
%% All of these layers, which are provided by the CSP and other
%% software vendors, need to be trusted, measured, and attested during the boot
%% process of the \ac{CVM}.
%% Moreover, a \ac{CVM} needs to be integrated into the cloud platform it is
%% running on, requiring the configuration of storage and networking, but also
%% additional components for monitoring by the cloud provider.
%
Given this large software stack running inside the \ac{CVM}, the issue of
trustworthiness comes again into focus. Effectively removing the infrastructure,
i.e., the \ac{CSP}, from the customer's \ac{TCB} requires
\begin{inparaenum}
\item an attestation infrastructure that allows to attest the authenticity of
  both the \ac{TEE} hardware and \emph{all} the software running inside it, and
\item transparency for software components provisioned by the \ac{CSP} in the
  \ac{CVM}, e.g., by supplying the underlying source code alongside a
  reproducible build process.
\end{inparaenum}

In this paper we examine \ac{CVM} offerings on public clouds along these two
axes in order to evaluate the level of trust in the \ac{CSP} these \ac{CC}
solutions still require.  In particular, we take a close look at the boot
process and provided attestion mechanisms involved in the setup of AMD SEV-SNP
\acp{CVM} on popular cloud providers. It turns out that there are shortcomings
that prevent achieving confidential computing under infrastructural threats. In
many cases this is due to missing attestation infrastructure or proprietary
software components that need to be included in the \ac{CVM}.
% , but a more
% thorough discussion of issues and potential solutions will be given below.
We make the following contributions:
%
\begin{itemize}
%
    \item We introduce a hierarchy of \emph{attestation levels} with
increasingly stronger guarantees for \acp{CVM} and showcase what could go
wrong with partial attestation at the lower levels of this hierarchy;
%
    \item We highlight that, if some components cannot be verified by the
    \ac{CVM} owner independently, some blind trust in the \ac{CSP} is still
    required when deploying services in commercial clouds;
%
    \item We conduct a case study into AMD SEV-SNP as provided by popular
commercial \acp{CSP} to assess how products meet our attestation levels, and we
explain shortcomings in current offerings.
%
\end{itemize}

